\begingroup
\let\clearpage\relax
\chapter{Projekt komputera 8-bitowego}
\section{Analiza układów wchodzących w skład komputera}
\subsection{Procesor MOS 6502}
Architektura procesora MOS 6502 oparta jest na architekturze von Neumanna, jest to rodzaj architektury, w którym dane programu, są przechowywane w tym samym miejscu co kod programu. Oznacza to, że w pamięci RAM możemy mieć zapisane czyste dane, na przykład zapisane adresy w stosie, a w kolejnych adresach kod programu który właśnie wykonujemy. Wewnętrzna struktura procesora jest zaprojektowana w sposób umożliwiający zrównoważoną wydajność pomimo posiadania tylko dwóch rejestrów, oraz co najważniejsze, jej budowa jest prosta, co zostało osiągnięte przez użycie zmniejszonej ilości tranzystorów w porównaniu do innych procesorów z tego okresu.
Główne komponenty procesora to między innymi:
\begin{itemize}
\item Rejestry
\item Jednostka arytmetyczno-logiczna
\item Jednostka zegara i pomiaru czasu
\item jednostka sterująca 
\end{itemize}
\hfill \break
\textbf{Rejestry} wewnętrzne rejestry, a więc komórki pamięci umieszczone wewnątrz procesora o rozmiarach 8 oraz 16 bitów zależnie od roli, jaką pełnią, służą do przechowywania  tymczasowych obliczeń, adresów pamięci oraz funkcji indeksowania. Procesor zawiara rejestry przedstawione w tabeli.

\begin{longtable}{|p{5cm}|p{11cm}|}
\hline
\textbf{Rejestr} & \textbf{Opis} \\ \hline
\endfirsthead
\hline
\textbf{Rejestr} & \textbf{Opis} \\ \hline
\endhead
\hline
\endfoot
\hline
\caption{Rejestry procesora i ich funkcje} \\
\endlastfoot

Akumulator (A) & Główny rejestr dla wykonywanych operacji arytmetycznych i logicznych. Jest bezpośrednio połączony z jednostką arytmetyczno-logiczną. \\ \hline
Rejestry indeksów (X i Y) & Rejestry pomocnicze używane w trybach adresowania indeksowanego i jako liczniki w procesach iteracyjnych. \\ \hline
Licznik programu (PC) & Jest to 16-bitowy rejestr, którego zadaniem jest ustalenie adresu pamięci, w którym znajduje się następna instrukcja, która ma zostać wykonana. Jego wartość jest zwiększana automatycznie podczas cyklu pobierania lub zmieniana przez odpowiednie instrukcje (np. JMP, JSR). \\ \hline
Wskaźnik stosu (S) & 8-bitowy rejestr wskazujący na wierzch stosu, którego adres oznacza pierwszą stronę pamięci, od adresu \$0100 do \$01FF. Strony pamięci zostaną opisane w temacie pamięci RAM. \\ \hline
Rejestr stanu procesora (P) & 8-bitowy rejestr zawierający wartości zależne od wyniku ostatnio wykonanej operacji, nazywane flagami. Bity w tym rejestrze przyjmują format "NV1BDIZ", gdzie każdy z bitów odpowiada innej fladze. Flagi w procesorze 6502 to:

\begin{itemize}
\item \textbf{N Flaga ujemności}  
Flaga ta zostanie ustawiona po każdej operacji arytmetycznej (gdy do któregokolwiek z rejestrów A, X lub Y wpisywana jest wartość).

\item \textbf{V Flaga przeładowania}  
Podobnie jak flaga ujemności, flaga ta jest przeznaczona do użycia z 8-bitowymi liczbami całkowitymi ze znakiem. Na flagę będą miały wpływ dodawanie i odejmowanie, instrukcje PLP, CLV i BIT oraz sygnał sprzętowy SO. Sygnał SO (Set Overflow) jest dostępny w procesorze 6502 (choć nie był dostępny we wszystkich procesorach z rodziny 65XX) i służy do ustawienia flagi V. Umożliwia to reakcję na działanie I/O w czasie równym lub krótszym niż trzy cykle zegara, gdy używana jest instrukcja BVC rozgałęziająca się do samej siebie (\$50 \$FE). Instrukcja CLV czyści flagę V, a instrukcje PLP i BIT kopiują wartość flagi z bitu 6 najwyższego wpisu stosu lub z pamięci. Po dodaniu lub odjęciu binarnym flaga V zostanie ustawiona w przypadku przepełnienia znaku, w przeciwnym razie zostanie wyczyszczona. 

\item \textbf{1 Flaga nieużywana} 
Zgodnie z obecnym stanem wiedzy, flaga ta ma zawsze wartość 1.

\end{itemize} \\ \hline \\
Rejestr stanu procesora (P) &
\begin{itemize}
\item \textbf{B Flaga przerwań} 
Flaga ta służy do rozróżniania przerwań programowych (BRK) od przerwań sprzętowych (IRQ lub NMI). Flaga B jest zawsze ustawiona, z wyjątkiem sytuacji, gdy rejestr P jest wypychany na stos podczas skoku do programu obsługi przerwań w celu przetworzenia tylko przerwania sprzętowego. Oficjalna dokumentacja NMOS 65xx twierdzi, że instrukcja BRK może powodować tylko skok do wektora IRQ (\$FFFE). Jeśli jednak podczas wykonywania instrukcji BRK wystąpi przerwanie NMI, procesor wykona skok do wektora NMI (\$FFFA), a rejestr P zostanie wepchnięty na stos z ustawioną flagą B.

\item \textbf{D Flaga trybu dziesiętnego} 
Flaga ta służy do wyboru trybu dziesiętnego (kodowanego binarnie) dla dodawania i odejmowania. W większości aplikacji flaga ta ma wartość zero.

\item \textbf{I Flaga wyłączenia przerwań} 
Flaga ta służy do zapobiegania przeskakiwaniu procesora do wektora obsługi IRQ (\$FFFE), gdy linia sprzętowa -IRQ jest aktywna. Flaga zostanie automatycznie ustawiona po odebraniu przerwania, dzięki czemu procesor nie będzie przeskakiwał do procedury obsługi przerwań, jeśli sygnał -IRQ pozostanie niski przez kilka cykli zegara.

\item \textbf{Z Flaga Zerowa}
Flaga Zero zostanie zmieniona w tych samych przypadkach, co flaga ujemna. Ogólnie rzecz biorąc, zostanie ona ustawiona, jeśli rejestr arytmetyczny jest ładowany z wartością zero, a wyczyszczona w przeciwnym razie. 

\end{itemize} \\ \hline \\
Rejestr stanu procesora (P) &
\begin{itemize}

\item \textbf{C Flaga przenoszenia}
Flaga ta jest używana w dodawaniu, odejmowaniu, porównywaniu i rotacji bitów. W dodawaniu i odejmowaniu działa jako 9 bit i pozwala na łączenie operacji w celu obliczenia liczb większych niż 8-bitowe. Podczas odejmowania flaga Carry jest ujemna w stosunku do Borrow: jeśli wystąpi przepełnienie, flaga zostanie wyczyszczona, w przeciwnym razie ustawiona. Porównania są szczególnym przypadkiem odejmowania: zakładają, że flaga Carry jest ustawiona, a flaga Decimal jest wyczyszczona i nie przechowują nigdzie wyniku odejmowania. Wszystkie z nich przechowują bit, który jest obracany do flagi przenoszenia. Instrukcje przesunięcia w lewo to ROL i ASL. ROL kopiuje początkową flagę Carry do najniższego bitu bajtu; ASL zawsze ją czyści. Podobnie, instrukcje ROR i LSR przesuwają w prawo.

\end{itemize} \\ \hline

\end{longtable}


Procesor 6502 zawiera kilka wewnętrznych rejestrów niezbędnych do wykonywania instrukcji i zarządzania danymi. Accumulator (A) jest głównym rejestrem używanym do operacji arytmetycznych i logicznych, współpracującym bezpośrednio z ALU. Rejestry indeksowe (X i Y) pomagają w adresowaniu indeksowym i działają jako liczniki podczas procesów iteracyjnych. Licznik programu (PC) to 16-bitowy rejestr, który śledzi adres pamięci następnej instrukcji do wykonania, automatycznie zwiększany podczas cyklu pobierania lub wyraźnie modyfikowany przez instrukcje sterujące, takie jak JMP. Wskaźnik stosu (SP) to 8-bitowy rejestr wskazujący na szczyt stosu, który znajduje się na stronie pamięci \$01 i ma kluczowe znaczenie dla obsługi wywołań podprogramów, przerwań i tymczasowego przechowywania danych. Wreszcie, Processor Status Register (P) to 8-bitowy rejestr, który przechowuje flagi odzwierciedlające wyniki operacji, w tym flagę Carry (C) dla przeniesień arytmetycznych lub operacji pożyczania, flagę Zero (Z) dla wyników równych zero, flagę Negative (N) odzwierciedlającą MSB w celu wskazania wartości ujemnych oraz flagę Overflow (V) sygnalizującą przepełnienie arytmetyczne ze znakiem.

Graficzne przedstawienie zależności między rejestrami, oraz polecenia służące do zmiany wartości w danym rejestrze prezentują się następująco:
\begin{figure}[h]
    \centering
    \includegraphics[width=0.8\textwidth]{images/Transfer_Operations.png} % Wstaw nazwę pliku z obrazem
    \caption{ \textit{Rejestry procesora MOS 6502}}
    \label{fig:Transfer_Opertaions} 
    \vspace{0.5em} % Odstęp między podpisem a źródłem
    \footnotesize Źródło: Opracowanie własne na podstawie \parencite{sander_cederlof_6502}
\end{figure}

\endgroup
